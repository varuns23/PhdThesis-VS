\begin{center}

{\large \textbf{Search for excited quarks at $\sqrt{s}=8\unit{TeV}$ with the CMS Experiment at the Large Hadron Collider}} \\
%
\vspace{0.5cm}
%
{\large \textbf{Varun Sharma}} \\
%
{\large \textbf{Department of Physics \& Astrophysics \\ University of Delhi}} \\
%
\vspace{0.4cm}
%
%%{\large \emph{THESIS ABSTRACT}}
{\large \emph{ABSTRACT}}
%\doublespacing
%{\Large \textbf{ ABSTRACT}}
%\vspace{0.5cm}
\end{center}
Quarks and leptons are understood to be fundamental particles within the framework of the standard model of particle physics. Their existence and
properties have been verified by different experiments. Despite its success, the standard model is far from being a complete theory. There are several 
unexplained phenomena which motivates us to go beyond the standard model. One of the directions is to search for the substructure of quarks, or, in other 
words, search for the compositeness of quarks. The motivation which makes us turn to this path comes from the experience of the past, in that the use of 
higher energies to probe the sub-structure, has often led to the observation of something more fundamental. The most compelling sign for the sub-structure 
of quarks would be the discovery of an excited state of a quark, often termed as the ``excited quarks'' and denoted by \qstar. Excited quarks may provide 
some insight as to why there exist three generations of quarks and leptons. Many different models of excited quarks have been put forward, where they 
can be seen either by contact interactions for scale of compositeness, $\Lambda\gg\sqrt{s}$ or by gauge interactions for $\Lambda\le\sqrt{s}$. 
Different experiments, in the past and present have searched for excited quarks with no success and have put lower bounds on the compositeness 
scale, $\Lambda$, and mass of the excited quark, \mqstar. 

This thesis presents a search for excited quarks by looking at a final state comprising of a photon and a high \pt jet using 19.7\fbinv of proton-proton 
collision data collected by the CMS experiment at the LHC at a center-of-mass energy of 8\unit{TeV}. This final state is also mimicked by several other 
standard model processes. The dominant contribution comes from the standard model \gamjet production, which forms 
an irreducibe background for this search. The second largest background comes from QCD dijet or multi-jet processes where one of the high \pt jet mimics an 
isolated photon inside the detector. The $W/Z+\gamma$ processes could also give a similar final state, but due to its small cross section, this background is 
negligible. The \qstar signal and dominant standard model backgrounds are simulated using \pythia event generator and are compared with the data. It is assumed 
throughout the study presented in this thesis that the LHC center-of-mass energy is larger than the compositeness scale, $\Lambda$ and excited quarks have a mass 
scale comparable to that of the dynamics of the new binding force, i.e., $\mqstar=\Lambda$. Events containing photons and jets with high transverse momentum are 
selected to search for a resonance peak in the invariant mass distribution 
of the photon and jet. The core of the search for the \qstar resonance is the measurement of invariant mass distribution of the \gamjet system and the estimation of 
the background. The background is evaluated using a parameterized fit function estimated using the data. It is advantageous to use a fit function as even though the
shape and normalization may agree between the data and the MC simulated background, there are still considerable theoretical and experimental uncertainties to be 
taken care of in the analysis. The methodology of smooth parameterization makes use of the fact that \gamjet background always produces a smooth and monotonically
decreasing spectrum. The various sources of systematic uncertainties affecting only the signal MC, such as jet energy resolution, photon energy resolution, jet 
energy calibration scale, photon energy calibration scale, and uncertainty on the integrated luminosity, are considered in the analysis.
The data are found to be consistent with the standard model predictions with no evidence of quark compositeness. 

The results are presented in terms of the expected and observed 95\% confidence level upper limits on the $\sigma\times\mathcal{B}$ for excited quarks, \qstar, as
a function of \mqstar for coupling multiplier, $f=1.0$ and $f=0.5$, where $\sigma$ is the production cross section and $\mathcal{B}$ is
the branching ratio for $\qstar\to\gamjet$. For \qstar signal, acceptance times efficiency has been found to range from 54\% to 58\% for \qstar masses from 1\unit{TeV}
to 4\unit{TeV}. The observed limits are found to be consistent with those expected in the absence of a signal. Excited quarks within the mass range 
$0.7<\mqstar<3.5\,(2.9)\unit{TeV}$ are excluded at 95\% confidence level for coupling multipliers $f=1.0\,(0.5)$. Based on the theoretical predictions for different
coupling strengths from 0.1 to 1.0 and the observed limits, we also present exclusion limits of the excited quark mass as a function of coupling strength. This is the
first such study performed at the LHC.

The work presented in the thesis entitled ``Search for excited quarks at $\sqrt{s}=8\unit{TeV}$ with the CMS Experiment at the Large Hadron Collider'' has been published
in Physics Letters B 738 (2014) 274-293.

